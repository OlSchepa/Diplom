\newsection
\section{Техническое задание}
\subsection{Основание для разработки}

Основанием для разработки программной системы учёта сервисного обслуживания сельскохозяйственной техники является задание на выпускную квалификационную работу приказ ректора ЮЗГУ от «» 2024 года № 0000-0 «Об утверждении тем выпускных квалификационных работ и руководителей выпускных квалификационных работ».
\subsection{Назначение разработки}

Программная система учёта сервисного обслуживания сельскохозяйственной техники предназначена для планирования и учета выполнения регламентных технических процедур обслуживания, что включает в себя: проведение планового техобслуживания, процедур предпродажной подготовки, гарантийного и послегарантийного ремонта, других процедур для поддержки работоспособности сельхозтехники.

Задачами данной разработки являются:

\begin{itemize}
	\item Создание учетной системы для фиксации всех выполненных сервисных работ на сельхозтехнике;
	\item Разработка бизнес-процессов для четкого контроля выполнения исполнителями всех необходимых процедур;
	\item Разработка системы упреждающего планирования технических обслуживаний гарантийной сельхозтехникиу;
	\item Создание реестра проведенных сервисных работ на каждой единице сельхозтехнике для отслеживания полного жизненного цикла техники и работ, проведенных на технике
	\item Разработка официальной документации, предоставляемой владельцам сельхозтехники при выполнении гарантийных сервисных работ;
	\item Разработка отчетов, позволяющих проконтролировать ход работ и дать итоговое заключение о состоянии конкретной единице сельхозтехники;
\end{itemize}

\subsection{Требования к программной системе}
\subsubsection{Требования к данным программной системы}

На рисунке 2.1 представлена типовая схема бизнес-процесса сервисного обслуживания единицы сельхозтехники на каждом этапе:
\begin{figure}[H]
	\centering
	\includegraphics[width=0.8\linewidth]{"images/shema"}
	\caption[Типовая схема бизнес-процесса сервисного обслуживания единицы сельхозтехники на каждом этапе]{Типовая схема бизнес-процесса сервисного обслуживания единицы сельхозтехники на каждом этапе}
	\label{fig:-shema}
\end{figure}

\subsubsection{Функциональные требования к программной системе}

В разрабатываемой учетной системе должны быть предусмотрено следующие виды Документов и Бизнес-процессов:
\begin{enumerate}
	\item Документ «Заявка на выполнение сервисных работ».
	Содержит реквизиты:
		\begin{itemize}
			\item Контрагент – организация или частное лицо, делающее заявку;
			\item Заявитель – сотрудник организации-заявителя;
			\item Телефон заявителя – телефон для связи с сотрудником;
			\item Техника требующая обслуживания;
			\item Серийный номер техники;
			\item Вид сервисных работ;
			\item Краткое описание проблемы заказчика;
		\end{itemize}
		
	\item Документ «Гарантийный ремонт».
	Содержит реквизиты: 
		\begin{itemize}
			\item Предмет согласования – тип Документ «Гарантийный ремонт»;
			\item Точки маршрута бизнес-процесса в соответствии со схемой;
			\item Задачи для исполнения сотрудником, соответствующем должности;
			\item Сотрудник - исполнитель задачи на каждой точки маршрута бизнес-процесса по занимаемой должности;
		\end{itemize}
\end{enumerate}

\paragraph{Вариант использования «Установка фильтров»}

Заинтересованные лица и их требования: пользователь желает выбрать определённую деталь путем установки фильтров по запчастям, технике, цене или серийному номеру.
Предусловие: Пользователь открывает каталог «Запчасти».
Постусловие: список перезагружается, отображаемые товары из истории, а также новые товары, добавляемые в список, соответствуют заданным критериям фильтрации.

Основной успешный сценарий:
\begin{enumerate}
	\item Пользователь открывает каталог «Запчасти».
	\item Пользователь выбирает критерии.
	\item Пользователь ищет интересующую его деталь фильтруя запчасти по наименованию, технике или серийному номеру.
	\item Система отображает профиль с подробной информацией о запчасти.
\end{enumerate}

\paragraph{Вариант использования «Поиск»}

Заинтересованные лица и их требования: пользователь желает произвести поиск запчастей, имеющих в названии одно или несколько ключевых слов.
Предусловие: Пользователь открывает каталог «Запчасти».
Постусловие: список перезагружается, при наличии товаров, соответствующих поисковому запросу в базе данных, будет отображен их список. Если товары отсутствуют, отображается пустой список.

Основной успешный сценарий:

\begin{enumerate}
	\item Пользователь открывает каталог «Запчасти».
	\item Пользователь вводит в строку поиска ключевые слова
	\item Пользователь просматривает полученный список.
\end{enumerate}

\paragraph{Вариант использования «Просмотр фотографий»}

Заинтересованные лица и их требования: пользователь желает просмотреть полноразмерные фотографии запчасти.
Предусловие: У пользователя открыт каталог запчастей.
Постусловие: Есть фотографии запчастей.

Основной успешный сценарий:
\begin{enumerate}
	\item Пользователь переходит на страницу запчасти.
	\item Пользователь наводит курсор мыши на фотографию запчасти и кликает на неё.
	\item Фотография открывается в изначальном размере.
\end{enumerate}

\paragraph{Вариант использования «Выбор способа оплаты»}

Заинтересованные лица и их требования: пользователь желает выбрать способ оплаты: наличными или переводом.
Предусловие: Пользователь выбирает способ оплаты.
Постусловие: Стоимость товаров полная.

Основной успешный сценарий:
\begin{enumerate}
	\item Пользователь выбирает способ оплаты.
	\item Система выставляет счёт на оплату.
	\item Пользователь выбирает оплату наличными.
	\item Пользователь приезжает за товаром и оплачивает его.
\end{enumerate}

\subsubsection{Требования пользователя к интерфейсу программной системы}

В программной системе должны присутствовать следующие требования к интерфейсу для взаимодействия с пользователем:

\begin{enumerate}
	\item Интерфейс интуитивно понятный и легкий в использовании даже без специальной подготовки. Интуитивные элементы управления, понятные названия функций и логическая структура меню.
	\item Для повышения производительности система должна иметь возможность быстро находить необходимую информацию. Это включает в себя удобную навигацию, быстрый поиск и доступ к ключевым функциям через горячие клавиши или быстрые ссылки.
	\item Интерфейс должен быть адаптивным к различным устройствам и разрешениям экрана, чтобы пользователи могли использовать систему на компьютерах, ноутбуках, планшетах, мобильных устройствах. Кроме того, интерфейс должен быть отзывчивым и быстрым, чтобы минимизировать время ожидания при выполнении операций.
	\item Интерфейс должен обеспечивать безопасность данных и защиту информации от несанкционированного доступа. Это включает в себя функции аутентификации, контроль доступа и шифрование данных.
	\item Наличие в интерфейсе возможности генерации отчётов и аналитических данных для мониторинга производительности, анализа трендов и принятия управленческих решений.
\end{enumerate}

\subsection{Нефункциональные требования к программной системе}
\subsubsection{Требования к архитектуре}
Система должна быть способна масштабироваться в зависимости от изменяющихся потребностей бизнеса. Это включает возможность добавления новых запчастей и техники, увеличения объема обрабатываемых данных и расширения функциональности без значительного изменения архитектуры.


\subsubsection{Требования к безопасности}

Система должна обеспечивать защиту данных и доступа к функциональности. Это включает в себя механизмы аутентификации и авторизации пользователей, шифрование данных в покое и в передаче, а также контроль доступа к различным частям системы.

\subsubsection{Требования к программному обеспечению}
Для реализации программной системы использовался язык программирования 1C. 

\subsubsection{Требования к надёжности}
Архитектура должна быть спроектирована с учётом возможности отказа отдельных компонентов или целых систем. Это включает в себя резервирование ресурсов, механизмы обнаружения и восстановления отказов, а также регулярное резервное копирование данных.

\subsubsection{Требования к аппаратному обеспечению}

Архитектура должна быть совместимой с существующими и будущими технологиями и платформами, чтобы обеспечить долгосрочную устойчивость и эволюцию системы.

\subsubsection{Требования к оформлению документации}

Разработка программной документации и программного изделия должна производиться согласно ГОСТ 19.102-77 и ГОСТ 34.601-90. Единая система программной документации.
